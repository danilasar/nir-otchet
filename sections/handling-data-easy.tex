\section{Обработка табличных данных. Часть 1}
\subsection{Условие задания}
Выполнить задание. Вариант: <<Найти сумму нечетных элементов с чётными порядковыми номерами. Вывести номера минимальных четных элементов>>.

Создать приложение для выполнения задания. Использовать элемент формы DataGridView. Диапазон $[a,b]$ означает, что $mas[i][j] >= a$ и $mas[i][j] <= b$

Приложение должно выполнять следующие действия:

\begin{enumerate}
\item Возможность удалять и добавлять строки таблицы. Проект не должен аварийно завершаться при удалении несуществующей таблицы.
\item Проверять ввод не числовых данных как в таблицу, так и в остальные текстовые поля (если есть в задании).
\item Если есть диапазон значений $[a,b]$, проверять, что $a < b$.
\item Заголовок формы должен отражать суть задания.
\item Все элементы формы должны быть внятно подписаны (кнопки подписаны, у тестового поля должно быть написано, для чего оно нужно и т. д.)
\item В коде должны быть комментарии и отступы (код должен быть легко читаем).
\item В коде программы все элементы формы должны быть переименованы (btnName -  для кнопок, lblName - для ссылок, txtName - для текстового поля и т.д.) Наименования должны быть понятными.
\item Приложение должно корректно работать (выводить ответ или ошибку с соответствующим сообщением). После вывода ошибок при вводе корректных данных поля ошибок должны очищаться.  
\end{enumerate}

\subsection{Вид формы в конструкторе}
Форма имеет вид:

\begin{figure}
\centering
\includegraphics[width=0.5\linewidth]{images/handling-data-easy/form.png}
\caption{Форма окна для задания <<Простые вычисления>>}
\label{handling-data-easy-form}
\end{figure}

\subsection{Таблица с описанием элементов формы}
Все элементы формы были переименованы для большей читаемости. В таблице \ref{tab:handling-data-easy-form} представлены все изменения.

\begin{table}
\centering
\begin{tabular}{|m{0.3\textwidth}|m{0.3\textwidth}|m{0.3\textwidth}|}
\hline
\textbf{Описание элементов формы} & \textbf{Список изменённых атрибутов} & \textbf{Новое значение атрибута} \\
\hline
\hline
Окно формы & Text & Обработка табличных данных \\
Верхняя метка заголовка & Name & taskLabel \\
Нижняя метка заголовка & Name & taskLabel2 \\
Таблица & Name & arrayGridView \\
Кнопка <<Подсчитать>> & Name & countButton \\
Кнопка <<Добавить элемент>> & Name & addButton \\
Кнопка <<Удалить последний элемент>> & Name & popButton \\
Метка <<Сумма>> & Name & sumOddLabel \\
Метка <<Номера>> & Name & minsLabel \\
Поле для суммы & Name & outputOddSumTextBox \\
Поле для номеров & Name & outputMinsTextBox \\

\hline
\end{tabular}
\caption{Значение атрибутов элементов в приложении для обработки табличных данных}
\label{tab:handling-data-easy-form}
\end{table}

\subsection{Примеры правильной и неправильной работы приложения}
При запуске приложения на экране появляется окно \ref{fig:handling-data-easy-start}.

\begin{figure}
\centering
\includegraphics[width=0.5\linewidth]{images//handling-data-easy/start.png}
\caption{Запуск программы}
\label{fig:handling-data-easy-start}
\end{figure}

При нажатии на кнопку подсчитываются и подставляются значения в поля вывода. Также происходит обработка ошибок.

\begin{figure}
\centering
\includegraphics[width=0.5\linewidth]{images//handling-data-easy/okay.png}
\caption{Запуск с корректными данными}
\label{fig:handling-data-easy-okay}
\end{figure}

\begin{figure}
\centering
\includegraphics[width=0.5\linewidth]{images//handling-data-easy/error.png}
\caption{Пример ввода с некорректными данными}
\label{fig:handling-data-easy-error}
\end{figure}

\begin{figure}
\centering
\includegraphics[width=0.5\linewidth]{images//handling-data-easy/error2.png}
\caption{Пример ввода с некорректными данными}
\label{fig:handling-data-easy-error2}
\end{figure}

\begin{figure}
\centering
\includegraphics[width=0.5\linewidth]{images//handling-data-easy/error3.png}
\caption{Пример ввода с некорректными данными}
\label{fig:handling-data-easy-error3}
\end{figure}

\subsection{Примеры исходного кода}
\begin{minted}{cpp}
	/* функция для обновления вывода */
	private: OutputState^ GetOutput() {
		long long sum = 0;
		long long minEl = LLONG_MAX;

		String^ columnName = "Массив";
		OutputState^ state = gcnew OutputState();

		bool oddsFound = false;
		bool ifSummable = false;
		bool evensFound = false;

		for (int i = 0; i < this->arrayGridView->RowCount; ++i) {
			long long val;
			Object^ cellValue = this->arrayGridView->Rows[i]->Cells[columnName]->Value;

			if (cellValue == nullptr) {
				continue;
			}
			try {
				val = Convert::ToInt64(cellValue);
				if (val % 2 != 0 && i % 2 == 0) {
					sum += Convert::ToInt64(cellValue);
					ifSummable = true;
					oddsFound = true;
				}
				else if (val % 2 != 0) oddsFound = true;
				else {
					evensFound = true;
					if (val < minEl) {
						minEl = val;
						state->resultIdx = Convert::ToString(i);
					}
					else if (val == minEl) {
						state->resultIdx += ", " + Convert::ToString(i);
					}
				}
			}
			catch (FormatException^ e) {
				state->resultSum = "Разрешены только целые числа";
				state->resultIdx = state->resultSum;
				return state;
			}
		}

		state->resultSum = Convert::ToString(sum);
		if (!oddsFound) {
			state->resultSum = "Нечётных элементов нет";
		}
		else if (!ifSummable) {
			state->resultSum = "На чётных индексах нет нечётных";
		}
		if (!evensFound) {
			state->resultIdx = "Чётных элементов нет";
		}
		return state;
	}
\end{minted}

Больше кода проекта доступно в приложении \ref{application-A}. Также в приложенном архиве можно найти полный код проекта.